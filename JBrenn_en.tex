%% start of file `template.tex'.
%% Copyright 2006-2010 Xavier Danaux (xdanaux@gmail.com).
%
% This work may be distributed and/or modified under the
% conditions of the LaTeX Project Public License version 1.3c,
% available at http://www.latex-project.org/lppl/.

% Version: 20110122-4


\documentclass[11pt,a4paper,nolmodern]{moderncv}

\usepackage{JBrenn}

\usepackage[english]{babel}
\linespread{0.9}
% for some reason, lines take up a lot of space in itemize in English...
\newenvironment{tightitemize}
   {\begin{itemize}
   \setlength{\parskip}{0pt}}
   {\end{itemize}} 

% personal data
\title{Junior Researcher | Eco-Hydrology}
\extrainfo{%
%\linkedin~
\httplink{orcid.org/0000-0002-6886-8792}\\%
%\octocat~
\httplink{www.github.com/JBrenn}\\%
Driving License} % optional, remove the line if not wanted

%\myquote{Freely you have received, freely give}{Matthew 10:8}


%\nopagenumbers{}                            % uncomment to suppress automatic page numbering for CVs longer than one page
%----------------------------------------------------------------------------------
%            content
%----------------------------------------------------------------------------------
\begin{document}
%\setmainfont{Minion Pro}
%\setsansfont{Myriad Pro}

\hyphenpenalty=10000
\maketitle

\section{Professional career}

\tlcventry{2014}{0}{Junior Research Assistent}{\href{http://www.eurac.edu/en/research/mountains/alpenv/Pages/default.aspx}{EURAC, Institute for Alpine} Environment}{Bozen/Bolzano}{}{eco-hydrological simulations with GEOtop and SWAT, eco-hydrologic climate impact studies at catchment scale, data management, organisation/realisation of field work (LTsER site Matsch/Mazia).}

\section{Education | Academic Training}

\tlcventry{2014}{0}{Seminars and Workshops}{}{}{}{e.g. School on Data analysis and programming with R (University of Bolzano, 3days), 
%INQUIMUS workshop: Challenges in Q2 methodologies to acquire and integrate data for the assessment of risk, vulnerability and resilience (EURAC research, 2days), 
Managing Open Science (EURAC research, 1day), Tri-Nationaler Workshop - Hydrologische Prozesse im Hochgebirge im Wandel der Zeit (Obergurgl, AT, 3days).}

\tldatecventry{2015}{SummerSchool}{\href{http://potsdam-summer-school.org/}{``Facing Natural Hazards''}}{}{Potsdam}{hosting institutions: \href{http://www.iass-potsdam.de/en}{IASS}, \href{https://www.pik-potsdam.de/pik-frontpage?set_language=en}{PIK}, \href{http://www.gfz-potsdam.de/en/home/}{GFZ}, \href{https://www.awi.de/en/about-us/sites/potsdam.html}{AWI}, \href{http://www.uni-potsdam.de/en/index.html}{University of Potsdam}}

\tldatecventry{2015}{MasterClass}{\href{http://www.euporias.eu/event/masterclass}{``Climate Services''}}{European Academy (EURAC)}{Bozen/Bolzano}{organized within the \href{http://www.euporias.eu/}{EUPORIAS project}.}
 
\tllabelcventry{2011}{2014}{2011--2014}{Scientific Assistent}{Institute of Earth and Environmental Science, Working Group Hydrology and Climatology}{University of Potsdam}{}{installation and maintenance of a micro-rain radar (S\"olden, Austria), contribution to an \href{http://www.hydrol-earth-syst-sci.net/17/863/2013/hess-17-863-2013.pdf}{open source library for processing weather radar data (wradlib)}, data analysis (R), literature review (Mendeley).}

\tllabelcventry{2011}{2014}{2011--2014}{Master's programme Geoecology}{Institute of Earth and Environmental Science}{University of Potsdam}{Graduation with distinction (1.2)}{Thesis: ``Spatial variability and temporal trends of soil moisture and evapotranspiration in an inneralpine dry catchment''. Supervised by Prof. Axel Bronstert \& Dr. Giacomo Bertoldi.} 
%(Institute for Alpine Environment, EURAC research, Bozen).}

\tldatecventry{2011}{Internship}{Potsdam Institute for Climate Impact Research (PIK)}{Potsdam}{}{Processing and analysis of climate and land use data as input for the Dynamic Global Vegetation Model LPJ with the statistic computing language \textit{R}. Supervised by Dr. Kirsten Thonike.}

\tldatecventry{2010}{Field Work}{Nacimiento}{Chile}{}{Determining soil hydraulic (e.g. hydraulic conductivity), terrain and vegetation characteristics in a forest catchment for parametrisation of the hydrological model WASA-Sed.}

\tldatecventry{2009}{Tutor ``mathematics for Geoecologists''}{Institute of Earth and Environmental Science}{University of Potsdam}{}{}

\tllabelcventry{2007}{2011}{2007--2011}{Bachelor's programme Geoecology}{Institute of Earth and Environmental Science}{University of Potsdam}{}
{Thesis: ``Modelling of sediment transport in a deforested catchment with \textit{WASA-Sed} (Nacimiento, Chile)'' Supervised by Prof. Axel Bronstert \& Dr. Christian Mohr (Institute of Earth and Environmental Science, University of Potsdam). Funded by DAAD scholarship ``Thesis abroad''.}


\newpage

\section{Skills}

\subsection{IT applications and development}

\cvcomputer{HydroModels}{GEOtop, SWAT, WASA-Sed}
		   {Geoinfo}{QGIS, GRASS, SAGA, ArcGIS}
		   
\cvcomputer{DataAnalysis}{R, Concave}
           {Programming}{Python, Fortran}

%\cvcomputer{Databases}{SQLite (RSQLite)}
%            {Formats}{NetCDF}

\cvcomputer{Operating Systems}{GNU/Linux (Ubuntu, RedHat), Windows}
		   {Tools}{GitHub, Mendeley, Overleaf}

\cvcomputer{Office}{LibreOffice, MicrosoftOffice, Inkscape}
           {Edition}{\LaTeX{}, Markdown}

\subsection{Foreign Languages}     
     
\cvlanguage{German}{Native}{Mother Tongue}
\cvlanguage{English}{Fluent}{Daily practice, scientific writing}
\cvlanguage{French}{B1 Level}{Studied 5 years in school}
\cvlanguage{Italian}{A2 Level}{Lived 3 years in South Tyrol (Italy)}


\section{Participation in Research Projects}

\tlcventry{2014}{0}{\href{http://www.monalisa-project.eu/}{MONALISA}}{\href{http://www.monalisa-project.eu/}{``Monitoring key evironmental parameters in the alpine environment involving science, technology and application''}}{}{}{Junior Researcher responsible for eco-hydrological modeling for different land uses (apple orchards, Alpine grassland - meadow and pasture), data management and analysis.}

\tlcventry{2014}{0}{\href{http://www.eurac.edu/en/research/mountains/remsen/projects/Pages/projectdetails.aspx?pid=11492}{HiResAlp}}{\href{http://www.eurac.edu/en/research/mountains/remsen/projects/Pages/projectdetails.aspx?pid=11492}{``An innovative framework for the integration of multi-source data to determine soil moisture and evapotranspiration at high resolution in Alpine regions''}}{}{}{Junior Researcher responsible for field activities and point-scale/distributed eco-hydrological modeling in the LTsER site Mazia/Matsch.}

\tllabelcventry{2013}{2014}{2013--2014}{\href{http://www.eurac.edu/en/research/mountains/alpenv/projects/Pages/projectdetails.aspx?pid=9221}{HydroAlp}}{\href{http://www.eurac.edu/en/research/mountains/alpenv/projects/Pages/projectdetails.aspx?pid=9221}{``Modelling the interactions between water cycle, vegetation and climate in Alpine Environments''}}{}{}
{Master Student responsible for climate change impact assessment with the hydrological model GEOtop, results available via \href{http://webgis.eurac.edu/hydroalp/}{WEB-GIS}.}

\section{Publications \& Conference Proceedings}

\tldatecventry{2015}{Bertoldi G, \underline{Brenner J}, Notarnicola C, Greifeneder F, Nicolini I, Della Chiesa S, Niedrist G, Tappeiner U}{Monitoring soil moisture patterns in alpine meadows using ground sensor networks and remote sensing techniques.}{Geophysical Research Abstracts, Vol. 17, 2015, European Geosciences Union, General Assembly 2015 - Vienna}{Austria, 12\,--\,17 April 2015 (poster)}{}

\tldatecventry{2014}{\underline{Brenner J}, Bertoldi G, Della Chiesa S, Niedrist G, Tappeiner U, Bronstert A}{Modellazione degli impatti del cambiamento climatico sulla distributione spaziale dell'evapotranspirazione, dell'umidita del terreno e del manto nevoso in una vallata alpina}{Atti del XXXIV Convegno Nazionale di Idraulica e Construzioni Idrauliche}{Bari, 07\,--\,10 Sept 2014}{}

\tldatecventry{2014}{\underline{Brenner J}, Bertoldi G, Della Chiesa S, Niedrist G, Tappeiner U, Bronstert A}{Modeling impacts of climate change on evapotranspiration and soil moisture spatial patterns in an alpine catchment}{Geophysical Research Abstracts, Vol. 16, 2014, European Geosciences Union, General Assembly 2014}{Vienna, Austria, 27 April\,--\,2nd May 2014 (talk)}{}

\tldatecventry{2012}{\underline{Brenner J}, Faul F, Ittner S, Scheiffele L, Schlingmann A, Vo\ss{} S, Wei\ss{}huhn P and Zoerner M (2012)}{\href{http://openlandscapes.zalf.de/openLandscapesWIKI_Glossaries/Ecosystem Services from Agro- and Forest Ecosystems.aspx}{Ecosystem Services from Agro- and Forest Ecosystems}}{openLandscape Wiki}{Leibniz Centre for Agricultural Landscape Research (ZALF)}{}

\section{References}
\cventry{1}{\href{http://www.geo.uni-potsdam.de/member-details/show/212.html}{Prof. Dr. Axel Bronstert}}{University of Potsdam}{Institute of Earth- and Enviromental Sciences}{\href{mailto:axelbron@uni-potsdam.de}{axelbron@uni-potsdam.de}}{}{}
\cventry{2}{\href{http://www.eurac.edu/en/research/mountains/alpenv/staff/Pages/staffdetails.aspx?persId=8583}{Dr. Giacomo Bertoldi}}{European Academy Bozen/Bolzano}{Institute for Alpine Environment}{\href{mailto:Giacomo.Bertoldi@eurac.edu}{Giacomo.Bertoldi@eurac.edu}}{}{}


\end{document}

